\section{Conclusion}
This study achieved its main goals of matching music selections to user's emotional states by effectively integrating music therapy and \gls{fer} through a web application.
It integrated advanced \gls{ml} techniques for real-time \gls{fer} and made use of the PERN stack for development.
Despite acknowledging certain limitations, such as scalability and the unfinished Youtube API integration, the project adapted to feedback effectively, focusing on four main emotions to simplify and enhance the therapeutic aspects of the application.
\\
\section{Future Work}
In order to improve the functionality of the application and user involvement, several initiatives are considered.
The user experience will be improved by implementing a user feedback system, which enable the generation of personalized and adaptable music playlists.
Enabling users to specify their nationality during sign-up will help with the suggestion of music that is culturally relevant, hence expanding the application's reach internationally.
\\
\indent Additionally, new features that let users remove their accounts and get their stored data will be implemented to ensure \gls{gdpr} compliance.\citep{gdpr_2018_general} 
Utilizing the emotional data gathered will also help the project by enhancing the \gls{fer} model's responsiveness and accuracy.
To dynamically modify music recommendation based on user interactions, it will be helpful to investigate advanced learning techniques such as reinforcement learning.
\\
\indent Lastly, a more thorough knowledge of users' emotional states will be possible by extending \gls{fer} capabilities to incorporate with physiological and behavioral indications such as speech patterns and brain signals \footnote{This includes \gls{eeg}, \gls{emg}, and \gls{eog} \citep{shin_2018_inner}}.
Besides these indicators, body movements and gestures are also helpful in achieving accurate recognition of an individual's emotions.

\section{Concluding Thoughts}
This project's multidisciplinary approach shows how cognitive science and technology  may be used to address complicated problem in health and wellbeing.
Although the application appears to have potential in matching music to identified emotional states, more testing is required to verify its effectiveness in improving emotional health. 
In addition to adding functionality to the application, the suggested future works aim to advance the field of emotion-sensitive interactive systems, which potentially offering benefits for users globally.
