% ------------------ PACKAGES ------------------ 
% Packages add extra commands and features to your LaTeX document. 
% In here, some of the most common packages for a thesis document have been added 

% LaTeX's float package
\usepackage{float}

\usepackage{multirow}

\usepackage{array}

% Manages hyperlinks 
\usepackage{url}
\usepackage[colorlinks=true,linkcolor=black,urlcolor=black,citecolor=black,breaklinks=true]{hyperref}
\usepackage[hyphens]{xurl} 
\urlstyle{same}

\usepackage{natbib}
\setcitestyle{aysep={,}}

% LaTeX's color package
\usepackage{color}

% LaTeX's main math package
\usepackage{amsmath}

% LaTeX's Caption and subcaption packages
\usepackage[format=plain,font=footnotesize,labelfont=bf]{caption}
\usepackage{subcaption}

% The graphicx package provides graphics support for adding pictures.
\usepackage{graphicx}

% Longtable allows you to write tables that continue to the next page.
\usepackage{longtable}

% The geometry packages defines the page layout (page dimensions, margins, etc)
\usepackage[a4paper, lmargin=2.5cm, rmargin=2.5cm, tmargin=2.5cm, bmargin=2.5cm]{geometry}

% Defines the Font of the document, e.g. Arimo font (Check Fonts here: https://tug.org/FontCatalogue/)
\usepackage[sfdefault]{arimo}

% Font encoding
\usepackage[T1]{fontenc}

% This package allows the user to specify the input encoding
\usepackage[utf8]{inputenc}

% This package allows you to add empty pages
\usepackage{emptypage}

% Allows inputs to be imported from a directory
\usepackage{import}

% Provides control over the typography of the Table of Contents, List of Figures and List of Tables
\usepackage{tocloft}

% The setspace package controls the line spacing properties.
\usepackage{setspace}

% Allows the customization of Latex's title styles
\usepackage{titlesec}

% Allows the customization of Latex's table of contents title styles
\usepackage{titletoc}

% The package provides functions that offer alternative ways of implementing some LATEX kernel commands
\usepackage{etoolbox}

% Provides extensive facilities for constructing and controlling headers and footers
\usepackage{fancyhdr} 

% Typographical extensions, namely character protrusion, font expansion, adjustment 
%of interword spacing and additional kerning
\usepackage{microtype}

% Generates PDF bookmarks
\usepackage{bookmark}

% Add color to Tables
\usepackage[table,xcdraw]{xcolor}

% Use these two packages together -- they define symbols
%  for e.g. units that you can use in both text and math mode.
\usepackage{gensymb}
\usepackage{textcomp}

% Acronym Package
\usepackage[acronym]{glossaries}

% Package for Matlab code
\usepackage{matlab-prettifier}

\usepackage{indentfirst}

% Change font size
\usepackage{relsize}

% Blindtext (only for template)
\usepackage{blindtext}

\usepackage{titlesec}

\usepackage{tabularx}

\usepackage{booktabs}

\usepackage{lipsum}

\usepackage[table]{xcolor}
\definecolor{lightgray}{gray}{0.9}

\usepackage{mdframed}

\usepackage{minted}
\usemintedstyle{friendly}

\usepackage{pdflscape}

\usepackage{siunitx} 
\sisetup{
  round-mode=places,    % Rounds numbers
  round-precision=2,    % to 2 places
  table-format=1.2,     % Assumes a number format of 1.22
  table-number-alignment=center, % Align numbers in table at center
  table-space-text-pre = (,      % Corrects spacing when numbers have parentheses
  table-space-text-post = ),
}

\usepackage{enumitem}
